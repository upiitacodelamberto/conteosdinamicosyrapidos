\documentclass[11pt]{article}
\usepackage[total={18cm,24cm},top=2cm,left=2cm]{geometry}
\begin{document}
\section*{Comentarios sobre la elaboraci\'on del programa}
El programa se empieza a desarrollar utilizando los datos del 
archivo 
$$
\mbox{\tt 1 Resultados$\_$secciones.csv} 
$$
(incluido en este mismo directorio).
\section*{Regla de Sturges}
La {\bf regla de Sturges}, propuesta por Herbert Sturges en 1926, es una 
regla pr\'actica acerca del n\'umero de clases que se deben considerar 
al elaborar un histograma. El n\'umero est\'a dado por la siguiente 
expresi\'on
$$
c=1+\mbox{log}_{2}M\mbox{, donde $M$ es el tama\~no de la muestra.}
$$
Que puede pasarse a logaritmo de base 10 de la siguiente forma:
$$
c=1+3.322\cdot\mbox{log}_{10}n
$$
siendo $n$ la cantidad de datos.\\
El valor de $c$ (n\'umero de clases) es com\'un redondearlo al entero 
m\'as cercano.
\section*{Determinaci\'on del n\'umero de clases}
Con el programa se obtuvo la siguiente informaci\'on:
\begin{verbatim}
Conteos din'amicos
Leyendo datos del archivo 1 Resultados_secciones.csv
Casilla  1  1	918	Votos=204
Casilla  2  1	918	Votos=204
Casilla  3  2	936	Votos=0
Casilla  4  2	936	Votos=0
Casilla  5  3	134	Votos=176
Casilla  6  3	134	Votos=176
Casilla  7  4	1506	Votos=162
Casilla  8  4	1506	Votos=162
Casilla  9  4	1548	Votos=218
Casilla 10  4	1548	Votos=213
Casilla 11  4	1551	Votos=177
Casilla 12  4	1551	Votos=177
Casilla 13  4	1552	Votos=234
Casilla 14  4	1552	Votos=234
Casilla 15  4	1239	Votos=233
Casilla 16  4	1519	Votos=244
Casilla 17  4	1519	Votos=244
Casilla 18  5	7	Votos=153
Casilla 19  5	7	Votos=153
Casilla 20  5	331	Votos=180
Casilla 21  5	331	Votos=181
Casilla 22  7	1300	Votos=213
Casilla 23  7	1300	Votos=212
Casilla 24  9	4690	Votos=257
Casilla 25  13	4546	Votos=154
Casilla 26  20	5536	Votos=207
Casilla 27  20	5546	Votos=207
Casilla 28  20	5546	Votos=207
Casilla 29  23	3531	Votos=71
Casilla 30  23	3531	Votos=72
Casilla 31  23	3551	Votos=184
Casilla 32  23	3551	Votos=184
Casilla 33  24	2497	Votos=204
Casilla 34  24	2497	Votos=204
Casilla 35  27	2809	Votos=261
Casilla 36  27	2809	Votos=261
Casilla 37  27	2809	Votos=251
Casilla 38  39	5545	Votos=8
Casilla 39  39	5545	Votos=8
\end{verbatim}
Como en el archivo {\tt 1 Resultados$\_$secciones.csv} hay informaci\'on de 39 casillas, 
el n\'umero de datos disponibles usando este archivo es de $n=39$. Por lo que se obtiene
$$
c=1+3.322\cdot\mbox{log}_{10}(39)=6.28\approx 6
$$
As\'i que se usar\'an 6 clases o bins. Como tenemos 39 casillas, y
$$
39=6\times 6 + 3
$$
Inicialmente se considerar\'an las clases 
conformadas por las casillas que se muestran a continuaci\'on
\begin{verbatim}
Clase       Casillas
  1          1 --  7
  2          8 -- 14
  3         15 -- 21 
  4         22 -- 28
  5         29 -- 35
  6         36 -- 39
\end{verbatim}

\end{document}
