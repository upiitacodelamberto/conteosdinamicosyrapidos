\documentclass{article}
\begin{document}
\section{Ruido Blanco}
Un proceso aleatorio Gaussiano estacionario se llama 
{\bf ruido blanco} cuando 
\begin{equation}
\langle n(t)\rangle = 0
\label{part2ec01}
\end{equation}
y
\begin{equation}
K_{n}(\tau)=\frac{N_{0}}{2}\delta(\tau)
\label{part2ec02}
\end{equation}
donde $\fracñ{N_{0}}{2}$ es una constante y $\delta()$ es la 
funci\'on delta de Dirac. Recordamos que $\delta()$ existe 
solamente en un punto
\end{document}
